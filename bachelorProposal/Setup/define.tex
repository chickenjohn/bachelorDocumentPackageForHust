\XeTeXlinebreaklocale "zh"
%\XeTeXlinebreakskip = 0pt plus 1pt minus 0.1pt
\geometry{top=25mm,bottom=20mm,left=30mm,right=30mm}%调整正文上下边距
\chead{\zhongsong 华中科技大学本科生毕业设计(论文)开题报告}                                                                                                
\renewcommand{\headrulewidth}{1pt}  %页眉线宽,设为0可以去页眉线
\lhead{}%页眉左边
\rhead{}%页眉右边
%\oddsidemargin=11mm
%\evensidemargin=5mm
\setCJKmainfont{SimSun}
\setCJKmonofont{SimSun}
\setmainfont{Times New Roman}
\fontsize{10.5pt}{15.7pt}\selectfont
\graphicspath{{Titlepage/}{Sect1/}{Sect2/}{Sect3/}{Sect4/}}
%\numberwithin{equation}{section}

\pagestyle{fancy}
%-------调整表格线条粗细-------
%\newcolumntype{L}[1]{>{\vspace{0.5em}\begin{minipage}{#1}\raggedright\let\newline\\
%\arraybackslash\hspace{0pt}}m{#1}<{\end{minipage}\vspace{0.5em}}}
%\newcolumntype{R}[1]{>{\vspace{0.5em}\begin{minipage}{#1}\raggedleft\let\newline\\
%\arraybackslash\hspace{0pt}}m{#1}<{\end{minipage}\vspace{0.5em}}}
%\newcolumntype{C}[1]{>{\vspace{0.5em}\begin{minipage}{#1}\centering\let\newline\\
%\arraybackslash\hspace{0pt}}m{#1}<{\end{minipage}\vspace{0.5em}}}
%\makeatletter
%\def\hlinewd#1{%
%  \noalign{\ifnum0=`}\fi\hrule \@height #1 \futurelet
%   \reserved@a\@xhline}
%\makeatother
%--------------------------------

% \def\degree{${}^{\circ}$}
% \def\ee{\mathrm{e}}%自然对数底数
% \def\ii{\mathrm{i}}%复数单位
% \def\dd{\mathrm{d}}
% \def\pp{\partial}
% \def\z{\left}
% \def\y{\right}
% \def\ol{\overline}
% \def\ora{$\overrightarrow{}$}

%----------字体相关----------
\setCJKfamilyfont{song}{SimSun}
\newcommand{\song}{\CJKfamily{song}} 
\setCJKfamilyfont{hei}{SimHei}
\newcommand{\heiti}{\CJKfamily{hei}}
\setCJKfamilyfont{zs}{STZhongsong}
\newcommand{\zhongsong}{\CJKfamily{zs}}
\setCJKfamilyfont{kai}{KaiTi}
\newcommand{\kaishu}{\CJKfamily{kai}} 
\newfontfamily\Rom{Times New Roman}
\setlength{\baselineskip}{0em}
\newcommand{\chuhao}{\fontsize{42pt}{\baselineskip}\selectfont}     %初号  
\newcommand{\xiaochuhao}{\fontsize{36pt}{\baselineskip}\selectfont} %小初号  
\newcommand{\yihao}{\fontsize{28pt}{\baselineskip}\selectfont}      %一号  
\newcommand{\erhao}{\fontsize{21pt}{\baselineskip}\selectfont}      %二号  
\newcommand{\xiaoerhao}{\fontsize{18pt}{\baselineskip}\selectfont}  %小二号  
\newcommand{\sanhao}{\fontsize{15.75pt}{\baselineskip}\selectfont}  %三号  
\newcommand{\sihao}{\fontsize{14pt}{18pt}\selectfont}%     四号  
\newcommand{\xiaosihao}{\fontsize{12pt}{18pt}\selectfont}  %小四号  
\newcommand{\wuhao}{\fontsize{10.5pt}{15.7pt}\selectfont}    %五号  
\newcommand{\xiaowuhao}{\fontsize{9pt}{13pt}\selectfont}   %小五号  
\newcommand{\liuhao}{\fontsize{7.875pt}{\baselineskip}\selectfont}  %六号  
\newcommand{\qihao}{\fontsize{5.25pt}{\baselineskip}\selectfont}    %七
%\newcommand{\bb}[1]{\raisebox{-2ex}[0pt][0pt]{\shortstack{#1}}}% 表格中向下移动文字2ex
\renewcommand{\abstractname}{\xiaoerhao \heiti \textbf{摘~~要}}
\renewcommand{\contentsname}{\xiaoerhao \heiti \textbf{目~~录}}
\renewcommand{\figurename}{\wuhao\kaishu 图}
\renewcommand{\tablename}{\wuhao\kaishu 表}
\renewcommand\refname{\sihao \song \textbf{参考文献}}
%\renewcommand{\thesubsection}{\Alph{subsection}}
\titleformat{\section}{\sihao \song \bf}{\thesection}{1em}{}
\titleformat{\subsection}{\xiaosihao \song \bf}{\thesubsection}{1em}{}
%\MakeRobust{\overrightarrow}%解决caption中使用向量的问题配合命令\usepackage{fixltx2e}
%\MakeRobust{\vv}
%
%\makeatletter
\def\fnum@figure#1{\figurename\nobreakspace\thefigure\hspace{1em}}% 去掉图后面的冒号并加空白空1em
\def\fnum@table#1{\tablename\nobreakspace\thetable\hspace{1em}}% 去掉表后面的冒号并加空白1em
%\makeatother

\hypersetup{colorlinks=false,pdfborder=0 0 0} 
%\setcounter{tocdepth}{2}

%\setlength{\droptitle}{-4em}     % Eliminate the default vertical space
%\addtolength{\droptitle}{-20mm}


