\section{绪论}
\subsection{研究背景}
\subsubsection{Android系统发展状况}

Android系统是由Google公司开发的一款面向智能手机、平板电脑等终端的操作系统。近年来,采用Android作为操作系统的手机持续占有80%以上的手机市场\cite{idc_research_inc_smartphone_????},而其中采用5.0及以上版本Android系统的手机占Android平台设备总量的43%\cite{statista_inc_android_????}。可以说,基于Android系统(尤其是Android5.0及以上版本系统)的智能设备已经广泛普及。因此,采用Android平台作为运行环境的应用程序能够获得较为广泛的使用。

作为一款智能设备操作系统,Android提供了对蓝牙(Bluetooth)通信、无线局域网(Wireless LAN,WLAN)通信和近场通信(Near Field Communication,NFC)等通信模式的底层驱动支持和应用程序接口支持,方便其与外部设备的沟通\cite{yaghmour_embedded_2013}。同时,采用Android系统的智能手机等设备的处理性能普遍高于一般的专用嵌入式设备,这也允许人们通过智能手机进行一些较为复杂的运算(如面部识别),减少了日常生活中的运算成本。通过Android平台,人们已经实现了对家用电器、个人电脑、显示器等设备的简单控制,同时一些简单的数据处理工作(如个人文档处理、面部识别、语音识别等)也能够借助Android平台实现。

与此同时,医疗设备的小型化、便携化成为了医学领域电子设备发展的趋势。由于 近年来移动终端计算能力的提高,医学设备的小型化、便携化程度逐步提高。更加轻便的医学设备既简化了医学检测方法,又方便了未经医学训练的普通人对自己身体状况的关注,在体育运动领域也体现出了潜在的应用价值\cite{moreno_android_2012}。在这种背景下,将医疗设备与Android设备结合是一个自然而然的想法:Android显然有助于在将医疗设备便捷化的同时,保证医疗设备所需的计算能力。而对于医疗设备中形式多样的生物传感器,Android设备也可选择使用蓝牙、WiFi、NFC等多种通信方式与之连接。目前在Android平台,已有多种健康类应用共用户选择下载或购买。

\subsubsection{心电图及其监测技术}

心电图(Electrocardiograph,ECG)反映了心脏功能的电信号活动记录,这些电信号由心脏中心肌细胞的除极和复极过程产生,并且可以通过在人体表面贴加电极(被称为导联)、测量不同部位的电势差而记录下来\cite{moreno_android_2012}\cite{_zhenduanxue_2013}。通过观察ECG信号的形态和一些关键点的数据,医生能够判定出病人的心脏功能存在哪些异常,而正常人心脏跳动所形成的ECG信号能够提供诸如心率等主要的人体体征指标。

由于ECG监测设备要求使用者长时间佩戴,因此便携性成为该类监测设备的发展方向之一。目前的医疗用便携式ECG监测设备均具有ECG信号记录、简单的信号显示和心率测量功能,经过记录的ECG数据往往存入SD Card中,需要分析时,将SD Card中的数据拷贝到计算机中进行分析。这些设备均配备了专用的低噪声电路,并且对信号进行了预处理,保证了ECG信号记录的有效性。还有部分便携式ECG设备也具有测量血氧饱和度(SpO2)的功能。目前使用的便携ECG监测仪主要用来完成长时间的ECG信号记录工作,并不能进行较为复杂的信号分析,且波形显示方法也缺乏交互性。

为了简化ECG数据的采集和处理过程、提高相关病情诊断效率,一些研究提出了借助智能手机平台处理ECG信号的设想。这些研究所提出的ECG监测仪大多由两部分组成:具备数据传输功能的可佩戴式ECG信号采集设备,以及涌来接收信号并加以处理的智能设备(比如一台Android智能手机)。已有研究证实了通过这种方法在Android设备上手机ECG等医疗数据的可行性,一些适用于Android平台的ECG信号处理算法也在验证后提出并应用。基于此,一些研究提出了专门处理ECG数据的Android平台应用这一构想。

\subsection{相关研究介绍及其局限}
\subsubsection{相关研究介绍}

已有研究主要着眼于两方面:基于Android平台ECG监测应用的开发,以及Android平台上对ECG信号的处理算法。

利用Android智能设备的蓝牙模块与ECG采集设备连接来获取数据、再加以处理,最后在Android端绘制波形的方案成为了相关研究的普遍选择。\cite{tang_android_2015}中的研究验证了Android设备上蓝牙连接的稳定性和速度:通过多线程技术实现了Android与最多7个蓝牙设备间的连接和数据传输,并指出在Android系统建立了4到7个基于RFCOMM协议的蓝牙连接后,传输速率保持在20KB/s到50KB/s之间,能够满足10KHz 16bit数据的传输。因此蓝牙连接完全能够满足两种设备间传输ECG信号的要求。参考文献\cite{moreno_android_2012}\cite{lou_wireless_2013}\cite{guo_ecg_2013}中的研究分别实现了Android设备对ECG信号的图形化显示。这些研究中的ECG显示模仿了ECG定走纸的形式,在Android设备屏幕上绘制出带有坐标轴的ECG信号图形,并且实现了多个导联信号的同时显示。研究\cite{guo_ecg_2013}还实现了心率波动波形的显示。

在ECG监测设备的功能性方面,许多研究利用Android平台的特点开发出了实用的功能。研究\cite{lou_wireless_2013}\cite{guo_ecg_2013}根据Android设备方便接入互联网的特点,实现了通过WiFi/3G网络将ECG信号或其他生理学指标传输到远端服务器或另一Android设备,供医生远程诊断的功能。研究\cite{guo_ecg_2013}中实现了ECG信号异常的警报功能,当警报触发时,应用能够自动输出长度为10s的ECG记录,并与医生的Android设备建立HTTP连接,传输图像。

同时,另一些研究尝试提出适合移动设备和可穿戴设备的信号分析算法。ECG信号十分微弱,容易受到外部环境和其他生理活动的干扰,对此一些研究提出了Android平台和可穿戴设备上无效ECG信号的识别方法,成功地对带噪声的ECG信号进行了识别和分类,并进行信号质量评估\cite{chudacek_simple_2011}\cite{satija_simple_2015}。研究\cite{oster_open_2013}提出了基于RR间期序列的心房颤动检测和基于机器学习的心室纤维性颤动检测算法。

\subsubsection{相关研究的局限性}

现有的ECG监测应用研究中,ECG波形在小尺寸屏幕(比如大多数Android智能手机的屏幕)上的显示不尽人意。由于手机屏幕无法完全显示标准的ECG波形,部分现有研究选择了等比例缩小ECG波形后完全显示,这样一来ECG波形的细节就无法方便观察;而使用横向屏幕显示完整比例的ECG波形时,留给用户进行操作的区域面积则减少了。同时,大部分研究在显示ECG波形时模拟了ECG定走纸的样式,这妨碍了数据的快速读取。

ECG信号处理算法方面,现有的算法着重解决某些疾病的自动检测及预警问题,已经能够有效检测出某些异常情况;另一些算法着重于实现移动设备和可穿戴设备上ECG信号的质量监测以及噪声消除\cite{satija_simple_2015}。但是现有的ECG监测应用并没有很好地将ECG信号的重要特征提取并表示出来,而是在后台获取ECG特征后判断是否需要发出预警。造成的结果是对于临床应用来说,现有ECG监测应用并不能有效地减少医护人员在分析ECG信号时的负担。

\subsection{本研究的成果和意义}

基于以往研究,本论文实现了Android 5.0及以上版本的ECG监测与记录应用,该应用能够通过蓝牙与ECG传感器连接,并创建基于SQLite的数据库,方便数据读取。同时该应用采用了两种波形显示方法,兼顾用户使用和波形细节的展示。在细节显示方面,提出一种触控标尺功能,解决了传统坐标中读取数据繁杂的问题。

在数据处理方面,提出一种新的基于局部最大值的心率检测方法,该方法能不受基线漂移和振幅突变的影响,准确地检测到ECG信号中R峰的位置;同时该方法具有迭代次数少、计算量小的特点,方便在嵌入式平台上实现。经测试,该算法的准确率为99.7\%,敏感度为99.7\%。

通过提出新的适用于Android端ECG应用的交互界面,本研究在专用于波形显示的交互界面设计上做出了新的尝试;而通过提出新的R峰探测方法, 本研究在ECG波形特征提取的关键步骤上做出了自己的探索。

\subsection{论文的组织结构}

本文内容按如下结构组织:

第一章介绍了医疗设备小型化的背景,总结出Android设备能够协助推进医疗设备小型化进程的观点,回顾了相关研究获得的成果、这些研究的关注点以及存在的不足,介绍了本研究获得的成果和其代表的意义;

第二章介绍了Android平台开发相关的背景技术,主要包括Android系统介绍、蓝牙技术在Android平台上的应用和其它Android平台支持的技术。这些技术在研究中均有涉及或实现;

第三章介绍了基于Android平台的ECG监测记录应用的设计、实现过程。从需求分析开始介绍了该应用完整的实现过程;

第四章介绍了应用中实现的心率检测算法及使用Octave进行算法建模的过程; 

第五章介绍了应用中相关算法的测试方法和测试结果,总结了应用的特点与不足,并对未来的相关工作进行了规划。

