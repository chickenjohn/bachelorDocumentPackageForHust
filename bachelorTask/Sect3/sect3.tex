\section{研究方案及研究计划}
\subsection{研究目的与方案}

本研究以实现Android平台的ECG监测和记录应用作为主要方向,意在于开发一个与ECG信号采集设备通过蓝牙连接,收取数据并且进行实时显示的Android平台应用。同时实现ECG中某些重要特征的提取算法,并将这些特征通过数据标签的形式显示出来,方便临床观察。同时,构建本地数据库存储接收到的ECG信号,方便查找与分析。

本研究计划实现Android智能设备与基于RFCOMM标准的蓝牙设备间传递数据的方法,并尝试将该方法应用于一种16-bit ECG数据的传递。之后,尝试构建一种数据类,在类中实现基于Android平台的SQLite数据库管理方法。在实现ECG数据的管理后尝试构建一种基于SurfaceView类的波形显示类,对存储的数据进行显示。同时探索基于数据库的ECG分析方法,其中着重探索ECG信号特征提取算法,并尝试将适用于Android平台的算法作为方法嵌入数据库类中。最后,尝试提出一种友好的ECG特征显示方法,并通过之前提出的波形显示类实现。

\subsection{研究计划}

本研究计划按四个主要研究方向划分,根据四个方向预计花费的时间安排计划表。项目的计划表见表\ref{tab1}。
\newpage

\begin{table}
\caption{\label{tab1}研究计划表}
\centering
\begin{tabular}{|p{0.4\textwidth}|p{0.5\textwidth}|}
\hline 
时间 & 研究内容 \\ 
\hline 
第1周~第3周 & 查找ECG信号处理、Android应用开发资料,翻译英文资料,撰写开题报告 \\ 
\hline 
第4周~第5周 & 熟悉Android Studio开发环境,熟悉各种ECG信号特征 \\ 
\hline 
第6周 & 设计、构建应用的UI Activity \\ 
\hline 
第7周~第8周 & 设计数据库与存储方法,构建蓝牙连接环境 \\ 
\hline 
第9周~第11周 & 构建波形显示类,实现ECG信号特征提取并显示 \\ 
\hline 
第12周~第14周 & 撰写毕业论文 \\ 
\hline 
第15周 & 答辩 \\ 
\hline 
\end{tabular} 

\end{table}